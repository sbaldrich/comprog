\documentclass[11pt, oneside]{article}   	% use "amsart" instead of "article" for AMSLaTeX format
\usepackage{geometry}                		% See geometry.pdf to learn the layout options. There are lots.
\geometry{a4paper}                   		% ... or a4paper or a5paper or ... 
%\geometry{landscape}                		% Activate for for rotated page geometry
\usepackage[parfill]{parskip}    		% Activate to begin paragraphs with an empty line rather than an indent
\usepackage{graphicx}	
\usepackage{fancyhdr}
\usepackage{minted}
\usepackage{amssymb}
\usepackage{enumerate}
\usepackage{relsize}
\usepackage{tocloft} %Those cute dots on TOC

\renewcommand{\cftsecleader}{\cftdotfill{\cftdotsep}}

\newmintedfile[cppcodefile]{cpp}{linenos=true,frame=leftline,framesep=2mm,tabsize=4}
\newminted{cpp}{linenos=true,frame=leftline,framesep=2mm}
\pagestyle{fancy}

\fancyhead{}
\fancyfoot{}
\renewcommand{\sectionmark}[1]{\markright{#1}}
\fancyhead[R]{\rightmark}
\fancyhead[L]{Competitive Programming Notebook : \textbf{Graph Theory}}
\renewcommand{\footrulewidth}{0.4pt}
\fancyfoot[R] {\thepage}


\title{Competitive Programming Notebook : Graph Theory}
\author{Santiago Baldrich}
\date{v1.0}
\begin{document}
\tableofcontents
\newpage

%:Start

\section{Articulation points}
\subsection{Articulation vertices}
Find articulation vertices in a \textbf{connected graph} using Tarjan's algorithm. 

\cppcodefile{../code/cpp/graphs/articulation.cpp}
\subsubsection{Field test: Networks }
\textbf{Source}:\textit{UVA 315}\\
A connected network is given. Find the number of articulation vertices.
\cppcodefile{../code/cpp/graphs/articulation_ex.cpp}

\section{Classic algorithms}
\subsection{Prim's minimum spanning tree}
\cppcodefile{../code/cpp/graphs/prim.cpp}
\subsection{Lowest common ancestor}
\cppcodefile{../code/cpp/graphs/lca.cpp}



\end{document}  













